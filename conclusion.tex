\sectionWithoutNumber{Заключэнне}

У дадзенай курсавой рабоце былі разгледзеныя асноўныя
прынцыпы метадалогіі Devops, бесперапыннай інтэграцыі,
бесперапыннай дастаўкі.
Разглядаліся перавагі папулярных інструментаў для рэалізацыі бесперапыннай
інтэграцыі і дастаўкі:
\begin{enumerate}
    \item Jenkins
    \item Travis CI
    \item GitLab CI
    \item Chef
    \item Puppet
    \item Better Code Hub
    \item Docker
    \item GitHub
\end{enumerate}

Таксама ў курсавой рабоце былі прааналізаваны перавагі ўкаранення
метадалогіі Devops, бесперапыннай інтэграцыі і дастаўкі ў працэс
распрацоўкі праграмнага забеспячэння.

Асноўнымі перавагамі ўкаранення метадалогіі Devops з'яўляюцца:
паляпшэнне культурнага асяроддзя ўнутры кампаніі,
пашырэнне кругагляду работнікаў на працэсы вытворчасці,
скарачэнне часу разгортвання,
павялічаная частата разгортвання,
памяншэнне выдаткаў на распрацоўкі і разгортванне праграмнага забеспячэння.
Канвееры бесперапыннай інтэграцыі і дастаўкі
прапаноўваюць памяншэнне рызыкі памылкі,
аўтаматызацыю працэсаў,
памяншэнне часу зваротнай сувязі на новы функцыі праграмы,
упэўненасць ў якасці праграмнага забеспячэння.

У апошнім раздзеле курсавой работы прыводзіўся прыклад пабудовы
канвеера бесперапыннай інтэграцыі і дастаўкі для Web-праграмы
пры дапамозе бясплатных сервісаў.
