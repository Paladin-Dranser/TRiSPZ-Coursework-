% Раздзел 1 — метадалогія devops
\section{Метадалогія DevOps}

\subsection{Асноўныя паняцці}

DevOps -- гэта культурны рух, які змяняе адносіны людзей да працы і
да яе вынікаў.\ref{book_effective_devops}

DevOps -- камбінацыя культуры, практык і інструментаў,
каторая павялічвае здольнасць арганізацыі пастаўляць праграмы і сервісы
з высокай хуткасцю: развіццё і паляпшэнне прадукцыі ў больш хуткім тэмпе,
чым арганізацыі, якія карыстаюцца традыцыйным спосабам распрацоўкі
праграмнага забеспячэння і
кіравання працэсамі інфраструктуры.\ref{site_aws.amazon.com/devops}

DevOps -- набор практык, накіраваны на актыўнае ўзаемадзеянне
спецыялістаў па распрацоўцы і спецыялістаў па інфармацыйна-тэхналагічнаму
абслугоўванню і ўзаемную інтэграцыю
іх працоўных працэсаў адно ў другое.\ref{site_ru.wikipedia.org/wiki/devops}

З вышэй прыведзеных азначэнняў тэрміна DevOps можам вызначыць,
што, хаця не існуе адзінага разумення метадалогіі DevOps,
у кожным выпадку значная ўвага надаецца культуры ўнутры арганізацыі,
узаемадзеянню спецыялістаў паміж сабой.
Дадзеная асаблівасць тлумачыцца тым, што метадалогія DevOps сфармавалася
на аснове гібкіх метадалогій, у якіх адбываецца пераход фокуса на
асобных людзей, узаемадзеянне і супрацоўніцтва.

Аднак рух DevOps пашырае прынцыпы гібкай распрацоўкі праграм
і прымяняе іх на ўзроўні арганізацыі ў цэлым,
у той час як іншыя гібкія метадалогіі акцэнтавалі ўвагу толькі на
распрацоўшчыках праграм.
