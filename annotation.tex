\thispagestyle{empty} % анатацыя не нумаруецца
\sectionWithoutContent{Анатацыя}
%\sectionWithoutContent{Аннотация}
%\sectionWithoutContent{Annotation}

У дадзенай курсавой рабоце расказваецца пра метадалогію Devops,
бесперапынную інтэграцыю і дастаўку.

У першым раздзеле разглядаліся асноўныя прынцыпы метадалогіі Devops,
перавагі ўкаранення Devops-практык.
У другім раздзеле разглядаліся асноўныя характарыстыкі
бесперапыннай інтэграцыі і дастаўкі, а таксама інструментаў
для іх рэалізацыі.
У трэцім раздзеле у якасці прыкладу быў пабудаваны
канвеер бесперапыннай інтэграцыі і дастаўкі для
Web-праграмы генерацыі фраз.

Курсавая работа змяшчае 25 старонак, 14 крыніц літаратуры,
1 табліцу, 10 малюнкаў, 5\,\,лістынгаў.

% русский язык

%В данной курсовой работе рассказывается о методологии Devops,
%непрерывной интеграции и доставке.

%В первом разделе рассмотрены основные принципы методологии Devops,
%преимущества внедрения Devops-практик.
%Во втором разделе рассмотрены основные характеристики
%непрерывной интеграции и доставки, а также инструментов
%для их реализации.
%В третьем разделе в качестве примера был построен
%конвейер непрерывной интеграции и доставки для
%Web-приложения генерации фраз.

%Курсовая работа содержит 25 страниц, 14 источников литературы,
%1 таблицу, 10 рисунков, 5 листингов.

% English

%The coursework describes Devops methodology,
%Сontinuous Integration and Delivery.

%The first section discusses the basic principles of the Devops methodology,
%the benefits of Devops practices.
%In the second section, the main characteristics of the
%Continuous Integration and Delivery, tools for their implementation.
%In the third section, describe an example of Continuous Integration
%and Delivery pipeline for Web-application to generate phrases.

%Coursework contains 25 pages, 14 sources of literature, 1 table,
%10 figures, 5 listings.
