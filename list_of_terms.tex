\thispagestyle{empty}
\sectionWithoutContent{Пералік умоўных адзнак, сімвалаў і тэрмінаў}
\noindent%
Agile -- метадалогія гібкай распрацоўкі.
\\
CI (Continuous integration) -- Бесперапынная інтэграцыя.
\\
CD (Continuous delivery) -- Бесперапынная дастаўка.
\\
RAD (Rapid Application Development) -- мадэль хуткай распрацоўкі.
\\
Канвеер (CI/CD-канвеер) -- зборны тэрмін, які аб'ядноўвае
стадыі распрацоўкі і праверкі якасці праграмнага забеспячэння
(кантроль версій кода, зборка, Unit-тэсты, разгортванне).
\\
Кантэйнер -- экземпляр кантэйнерызацыі, які змяшчае
неабходную праграму і канфігурацыю асяроддзя для яе правільнай
работы.
\\
Кантэйнерызацыя -- метад віртуалізацыі, пры якім ядро
аперацыйнай сістэмы падтрымлівае некалькі ізаляваных
экземпляраў прасторы карыстальнікаў замест аднаго[\ref{site:OS-level}].
\\
Інтэграцыя -- аб'ядноўванне некалькіх працоўных галін рэпазіторыя ў
галоўную галіну.
\\
Дастаўка -- выкладанне праграмнага забеспячэння на сервер
альбо іншае месца, адкуль яго можа ўстанавіць альбо абнавіць спажывец
праграмнага забеспячэння.
\\
Разгортванне -- канфігурацыя асяроддзя і ўстаноўка праграмнага
забеспячэння.
\\
Канфігурацыя -- настройка параметраў асяроддзя, ў якім будзе
працаваць праграмнае забеспячэння, і ўстаноўка неабходных элементаў
(ад якіх залежыць выкананне праграмы) для правільнага выканання
праграмным забеспячэннем сваіх функцый.
\\
Зборка -- працэс атрымання праграмнага забеспячэння з зыходнага коду.
\\
Распрацоўка -- дзейнасць па стварэнню праграмнага забеспячэння.
\\
Аднаўленне сістэмы -- аднаўленне працаздольнасці праграмнага
забеспячэння пры дапамозе вяртання на папярэднюю версію праграмы.
\\
Памылка -- памылка ў праграме альбо сістэме, якая змяняе
вынікі работы праграмы з чаканых на непрадказальныя.

\clearpage
