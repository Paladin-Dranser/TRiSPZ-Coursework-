%Прадмова
\sectionWithoutNumber{\prefacename}
З паскарэннем тэмпаў распрацоўкі праграмнага забеспячэння і
неабходнасці падстройвацца да дынамічных змен
развіваліся новыя метадолігіі распрацоўкі: вадаспад (Waterfall),
метадалогія гібкай распрацоўкі (Agile), спіральны метад (Spiral),
мадэль хуткай распрацоўкі (Rapid Application Development), devops і іншыя.


Devops — сучасная метадалогія распрацоўкі праграмнага забеспячэння, якая
дапамагае павялічыць прадукцыйнасць працы на ўзроўні арганізацыі.

Метадалогія devops шырока распаўсюджана сярод IT-кампаній замежный краін,
і ўжо паспела даказаць паспяховаць укаранення пэўных практык devops
у працэсы арганізацыі.

У той жа час у краінах на постсавецкіх тэрыторыях дадзеная метадалогія
вывучана слаба, што запавольвае яе распаўсюджванне, а таксама спараджае
разнастайныя непаразуменні.
Нягледзячы на гэта дадзеная метадалогія таксама пачынае ўкараняцца сярод
некаторых айчынных кампаній.

Дадзеная курсавая работа разглядае што з сябе прадстаўляе метадалогія
devops, паняцці бесперапынная інтэграцыя (CI) і бесперапынная дастаўка (CD),
іх узаемадзеянне і спосабы рэалізацыі.

\clearpage
