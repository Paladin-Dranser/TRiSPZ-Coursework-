%Прадмова
\sectionWithoutNumber{\prefacename}
З паскарэннем тэмпаў распрацоўкі праграмнага забеспячэння і
неабходнасці падстройвацца да дынамічных змен
развіваліся новыя метадолігіі распрацоўкі: вадаспад (Waterfall),
метадалогія гібкай распрацоўкі (Agile), спіральны метад (Spiral),
мадэль хуткай распрацоўкі (Rapid Application Development), devops і іншыя.


Devops — сучасная метадалогія распрацоўкі праграмнага забеспячэння, якая
дапамагае павялічыць прадукцыйнасць працы на ўзроўні арганізацыі.

Практыка devops шырока распаўсюджана сярод IT-кампаній замежный краін.
У той жа час дадзеная метадалогія таксама пачынае ўкараняцца сярод айчынных
кампаній.

Дадзеная курсавая работа разглядае метадалогію devops, паняцці
бесперапынная інтэграцыя (CI), бесперапынная дастаўка (CD),
іх узаемадзеянне і спосабы рэалізацыі.

\clearpage
